% Generated by Sphinx.
\def\sphinxdocclass{report}
\documentclass[letterpaper,10pt,english]{sphinxmanual}
\usepackage[utf8]{inputenc}
\DeclareUnicodeCharacter{00A0}{\nobreakspace}
\usepackage{cmap}
\usepackage[T1]{fontenc}
\usepackage{babel}
\usepackage{times}
\usepackage[Bjarne]{fncychap}
\usepackage{longtable}
\usepackage{sphinx}
\usepackage{multirow}
\usepackage{eqparbox}


\addto\captionsenglish{\renewcommand{\figurename}{Fig. }}
\addto\captionsenglish{\renewcommand{\tablename}{Table }}
\SetupFloatingEnvironment{literal-block}{name=Listing }



\title{SRT procedures Documentation}
\date{October 19, 2017}
\release{1}
\author{E. Egron, M. Pilia, S. Loru}
\newcommand{\sphinxlogo}{}
\renewcommand{\releasename}{Release}

\makeindex

\makeatletter
\def\PYG@reset{\let\PYG@it=\relax \let\PYG@bf=\relax%
    \let\PYG@ul=\relax \let\PYG@tc=\relax%
    \let\PYG@bc=\relax \let\PYG@ff=\relax}
\def\PYG@tok#1{\csname PYG@tok@#1\endcsname}
\def\PYG@toks#1+{\ifx\relax#1\empty\else%
    \PYG@tok{#1}\expandafter\PYG@toks\fi}
\def\PYG@do#1{\PYG@bc{\PYG@tc{\PYG@ul{%
    \PYG@it{\PYG@bf{\PYG@ff{#1}}}}}}}
\def\PYG#1#2{\PYG@reset\PYG@toks#1+\relax+\PYG@do{#2}}

\expandafter\def\csname PYG@tok@gd\endcsname{\def\PYG@tc##1{\textcolor[rgb]{0.63,0.00,0.00}{##1}}}
\expandafter\def\csname PYG@tok@gu\endcsname{\let\PYG@bf=\textbf\def\PYG@tc##1{\textcolor[rgb]{0.50,0.00,0.50}{##1}}}
\expandafter\def\csname PYG@tok@gt\endcsname{\def\PYG@tc##1{\textcolor[rgb]{0.00,0.27,0.87}{##1}}}
\expandafter\def\csname PYG@tok@gs\endcsname{\let\PYG@bf=\textbf}
\expandafter\def\csname PYG@tok@gr\endcsname{\def\PYG@tc##1{\textcolor[rgb]{1.00,0.00,0.00}{##1}}}
\expandafter\def\csname PYG@tok@cm\endcsname{\let\PYG@it=\textit\def\PYG@tc##1{\textcolor[rgb]{0.25,0.50,0.56}{##1}}}
\expandafter\def\csname PYG@tok@vg\endcsname{\def\PYG@tc##1{\textcolor[rgb]{0.73,0.38,0.84}{##1}}}
\expandafter\def\csname PYG@tok@vi\endcsname{\def\PYG@tc##1{\textcolor[rgb]{0.73,0.38,0.84}{##1}}}
\expandafter\def\csname PYG@tok@mh\endcsname{\def\PYG@tc##1{\textcolor[rgb]{0.13,0.50,0.31}{##1}}}
\expandafter\def\csname PYG@tok@cs\endcsname{\def\PYG@tc##1{\textcolor[rgb]{0.25,0.50,0.56}{##1}}\def\PYG@bc##1{\setlength{\fboxsep}{0pt}\colorbox[rgb]{1.00,0.94,0.94}{\strut ##1}}}
\expandafter\def\csname PYG@tok@ge\endcsname{\let\PYG@it=\textit}
\expandafter\def\csname PYG@tok@vc\endcsname{\def\PYG@tc##1{\textcolor[rgb]{0.73,0.38,0.84}{##1}}}
\expandafter\def\csname PYG@tok@il\endcsname{\def\PYG@tc##1{\textcolor[rgb]{0.13,0.50,0.31}{##1}}}
\expandafter\def\csname PYG@tok@go\endcsname{\def\PYG@tc##1{\textcolor[rgb]{0.20,0.20,0.20}{##1}}}
\expandafter\def\csname PYG@tok@cp\endcsname{\def\PYG@tc##1{\textcolor[rgb]{0.00,0.44,0.13}{##1}}}
\expandafter\def\csname PYG@tok@gi\endcsname{\def\PYG@tc##1{\textcolor[rgb]{0.00,0.63,0.00}{##1}}}
\expandafter\def\csname PYG@tok@gh\endcsname{\let\PYG@bf=\textbf\def\PYG@tc##1{\textcolor[rgb]{0.00,0.00,0.50}{##1}}}
\expandafter\def\csname PYG@tok@ni\endcsname{\let\PYG@bf=\textbf\def\PYG@tc##1{\textcolor[rgb]{0.84,0.33,0.22}{##1}}}
\expandafter\def\csname PYG@tok@nl\endcsname{\let\PYG@bf=\textbf\def\PYG@tc##1{\textcolor[rgb]{0.00,0.13,0.44}{##1}}}
\expandafter\def\csname PYG@tok@nn\endcsname{\let\PYG@bf=\textbf\def\PYG@tc##1{\textcolor[rgb]{0.05,0.52,0.71}{##1}}}
\expandafter\def\csname PYG@tok@no\endcsname{\def\PYG@tc##1{\textcolor[rgb]{0.38,0.68,0.84}{##1}}}
\expandafter\def\csname PYG@tok@na\endcsname{\def\PYG@tc##1{\textcolor[rgb]{0.25,0.44,0.63}{##1}}}
\expandafter\def\csname PYG@tok@nb\endcsname{\def\PYG@tc##1{\textcolor[rgb]{0.00,0.44,0.13}{##1}}}
\expandafter\def\csname PYG@tok@nc\endcsname{\let\PYG@bf=\textbf\def\PYG@tc##1{\textcolor[rgb]{0.05,0.52,0.71}{##1}}}
\expandafter\def\csname PYG@tok@nd\endcsname{\let\PYG@bf=\textbf\def\PYG@tc##1{\textcolor[rgb]{0.33,0.33,0.33}{##1}}}
\expandafter\def\csname PYG@tok@ne\endcsname{\def\PYG@tc##1{\textcolor[rgb]{0.00,0.44,0.13}{##1}}}
\expandafter\def\csname PYG@tok@nf\endcsname{\def\PYG@tc##1{\textcolor[rgb]{0.02,0.16,0.49}{##1}}}
\expandafter\def\csname PYG@tok@si\endcsname{\let\PYG@it=\textit\def\PYG@tc##1{\textcolor[rgb]{0.44,0.63,0.82}{##1}}}
\expandafter\def\csname PYG@tok@s2\endcsname{\def\PYG@tc##1{\textcolor[rgb]{0.25,0.44,0.63}{##1}}}
\expandafter\def\csname PYG@tok@nt\endcsname{\let\PYG@bf=\textbf\def\PYG@tc##1{\textcolor[rgb]{0.02,0.16,0.45}{##1}}}
\expandafter\def\csname PYG@tok@nv\endcsname{\def\PYG@tc##1{\textcolor[rgb]{0.73,0.38,0.84}{##1}}}
\expandafter\def\csname PYG@tok@s1\endcsname{\def\PYG@tc##1{\textcolor[rgb]{0.25,0.44,0.63}{##1}}}
\expandafter\def\csname PYG@tok@ch\endcsname{\let\PYG@it=\textit\def\PYG@tc##1{\textcolor[rgb]{0.25,0.50,0.56}{##1}}}
\expandafter\def\csname PYG@tok@m\endcsname{\def\PYG@tc##1{\textcolor[rgb]{0.13,0.50,0.31}{##1}}}
\expandafter\def\csname PYG@tok@gp\endcsname{\let\PYG@bf=\textbf\def\PYG@tc##1{\textcolor[rgb]{0.78,0.36,0.04}{##1}}}
\expandafter\def\csname PYG@tok@sh\endcsname{\def\PYG@tc##1{\textcolor[rgb]{0.25,0.44,0.63}{##1}}}
\expandafter\def\csname PYG@tok@ow\endcsname{\let\PYG@bf=\textbf\def\PYG@tc##1{\textcolor[rgb]{0.00,0.44,0.13}{##1}}}
\expandafter\def\csname PYG@tok@sx\endcsname{\def\PYG@tc##1{\textcolor[rgb]{0.78,0.36,0.04}{##1}}}
\expandafter\def\csname PYG@tok@bp\endcsname{\def\PYG@tc##1{\textcolor[rgb]{0.00,0.44,0.13}{##1}}}
\expandafter\def\csname PYG@tok@c1\endcsname{\let\PYG@it=\textit\def\PYG@tc##1{\textcolor[rgb]{0.25,0.50,0.56}{##1}}}
\expandafter\def\csname PYG@tok@o\endcsname{\def\PYG@tc##1{\textcolor[rgb]{0.40,0.40,0.40}{##1}}}
\expandafter\def\csname PYG@tok@kc\endcsname{\let\PYG@bf=\textbf\def\PYG@tc##1{\textcolor[rgb]{0.00,0.44,0.13}{##1}}}
\expandafter\def\csname PYG@tok@c\endcsname{\let\PYG@it=\textit\def\PYG@tc##1{\textcolor[rgb]{0.25,0.50,0.56}{##1}}}
\expandafter\def\csname PYG@tok@mf\endcsname{\def\PYG@tc##1{\textcolor[rgb]{0.13,0.50,0.31}{##1}}}
\expandafter\def\csname PYG@tok@err\endcsname{\def\PYG@bc##1{\setlength{\fboxsep}{0pt}\fcolorbox[rgb]{1.00,0.00,0.00}{1,1,1}{\strut ##1}}}
\expandafter\def\csname PYG@tok@mb\endcsname{\def\PYG@tc##1{\textcolor[rgb]{0.13,0.50,0.31}{##1}}}
\expandafter\def\csname PYG@tok@ss\endcsname{\def\PYG@tc##1{\textcolor[rgb]{0.32,0.47,0.09}{##1}}}
\expandafter\def\csname PYG@tok@sr\endcsname{\def\PYG@tc##1{\textcolor[rgb]{0.14,0.33,0.53}{##1}}}
\expandafter\def\csname PYG@tok@mo\endcsname{\def\PYG@tc##1{\textcolor[rgb]{0.13,0.50,0.31}{##1}}}
\expandafter\def\csname PYG@tok@kd\endcsname{\let\PYG@bf=\textbf\def\PYG@tc##1{\textcolor[rgb]{0.00,0.44,0.13}{##1}}}
\expandafter\def\csname PYG@tok@mi\endcsname{\def\PYG@tc##1{\textcolor[rgb]{0.13,0.50,0.31}{##1}}}
\expandafter\def\csname PYG@tok@kn\endcsname{\let\PYG@bf=\textbf\def\PYG@tc##1{\textcolor[rgb]{0.00,0.44,0.13}{##1}}}
\expandafter\def\csname PYG@tok@cpf\endcsname{\let\PYG@it=\textit\def\PYG@tc##1{\textcolor[rgb]{0.25,0.50,0.56}{##1}}}
\expandafter\def\csname PYG@tok@kr\endcsname{\let\PYG@bf=\textbf\def\PYG@tc##1{\textcolor[rgb]{0.00,0.44,0.13}{##1}}}
\expandafter\def\csname PYG@tok@s\endcsname{\def\PYG@tc##1{\textcolor[rgb]{0.25,0.44,0.63}{##1}}}
\expandafter\def\csname PYG@tok@kp\endcsname{\def\PYG@tc##1{\textcolor[rgb]{0.00,0.44,0.13}{##1}}}
\expandafter\def\csname PYG@tok@w\endcsname{\def\PYG@tc##1{\textcolor[rgb]{0.73,0.73,0.73}{##1}}}
\expandafter\def\csname PYG@tok@kt\endcsname{\def\PYG@tc##1{\textcolor[rgb]{0.56,0.13,0.00}{##1}}}
\expandafter\def\csname PYG@tok@sc\endcsname{\def\PYG@tc##1{\textcolor[rgb]{0.25,0.44,0.63}{##1}}}
\expandafter\def\csname PYG@tok@sb\endcsname{\def\PYG@tc##1{\textcolor[rgb]{0.25,0.44,0.63}{##1}}}
\expandafter\def\csname PYG@tok@k\endcsname{\let\PYG@bf=\textbf\def\PYG@tc##1{\textcolor[rgb]{0.00,0.44,0.13}{##1}}}
\expandafter\def\csname PYG@tok@se\endcsname{\let\PYG@bf=\textbf\def\PYG@tc##1{\textcolor[rgb]{0.25,0.44,0.63}{##1}}}
\expandafter\def\csname PYG@tok@sd\endcsname{\let\PYG@it=\textit\def\PYG@tc##1{\textcolor[rgb]{0.25,0.44,0.63}{##1}}}

\def\PYGZbs{\char`\\}
\def\PYGZus{\char`\_}
\def\PYGZob{\char`\{}
\def\PYGZcb{\char`\}}
\def\PYGZca{\char`\^}
\def\PYGZam{\char`\&}
\def\PYGZlt{\char`\<}
\def\PYGZgt{\char`\>}
\def\PYGZsh{\char`\#}
\def\PYGZpc{\char`\%}
\def\PYGZdl{\char`\$}
\def\PYGZhy{\char`\-}
\def\PYGZsq{\char`\'}
\def\PYGZdq{\char`\"}
\def\PYGZti{\char`\~}
% for compatibility with earlier versions
\def\PYGZat{@}
\def\PYGZlb{[}
\def\PYGZrb{]}
\makeatother

\renewcommand\PYGZsq{\textquotesingle}

\begin{document}

\maketitle
\tableofcontents
\phantomsection\label{index::doc}


Introduction TBD...

(Selection observation mode... follow the instruction...
Observation mode \textgreater{} Receiver \textgreater{} Backend...
How to start the observation, check during the session...)


\chapter{Observation modes}
\label{index:observation-modes}\label{index:welcome-to-srt-procedure-documentation}

\section{Continuum}
\label{Continuum/index::doc}\label{Continuum/index:continuum}

\subsection{C-band}
\label{Continuum/C-band/index::doc}\label{Continuum/C-band/index:c-band}

\subsubsection{Total Power}
\label{Continuum/C-band/TP/index:total-power}\label{Continuum/C-band/TP/index::doc}

\paragraph{Before observing}
\label{Continuum/C-band/TP/before-obs:before-observing}\label{Continuum/C-band/TP/before-obs::doc}
Some checks need to be performed before starting the observations.


\subparagraph{On nuraghe-mng}
\label{Continuum/C-band/TP/before-obs:on-nuraghe-mng}\begin{description}
\item[{Check that :}] \leavevmode\begin{itemize}
\item {} 
all of the 31 containers are active on ACS ;

\item {} 
the active surface is green on AS ;

\item {} 
the jlog is open in order to track possible error messages ;

\item {} 
the interface of the Meteo client is open to check the wind velocity in real time (\textless{} 60 km/h).

\end{itemize}

\end{description}


\subparagraph{On nuraghe-obs1}
\label{Continuum/C-band/TP/before-obs:on-nuraghe-obs1}\begin{enumerate}
\item {} 
Check the presence of the 8 panels :
\begin{itemize}
\item {} 
\textbf{operatorInput}

\item {} 
\textbf{AntennaBoss}

\item {} 
\textbf{GenericBackend}

\item {} 
\textbf{Mount}

\item {} 
\textbf{Observatory}

\item {} 
\textbf{Receivers}

\item {} 
\textbf{Scheduler}

\item {} 
\textbf{MinorServo}

\end{itemize}

\end{enumerate}
\begin{enumerate}
\setcounter{enumi}{1}
\item {} 
Upload your shedules (.scd, .lis, .bck and .cfg files) and check them :

\emph{From your computer:}

\code{\$ scp  {[}schedulename.*{]} observer@nuraghe-obs1:/archive/schedules/{[}projectID{]}}

\emph{On nuraghe-obs1:}

\code{\$ cd /archive/schedules/{[}projectID{]}}

\code{\$ scheduleChecker {[}schedulename.scd{]}}

\end{enumerate}


\paragraph{Start the observations}
\label{Continuum/C-band/TP/start-obs:start-the-observations}\label{Continuum/C-band/TP/start-obs::doc}
\$ : commands to insert in a shell

\textgreater{} : commands to insert in the operatorInput panel
\begin{enumerate}
\item {} 
Insert your project number
\begin{quote}

\code{\textgreater{} project={[}projectID{]}}
\end{quote}

\item {} 
Initial setup
\begin{quote}

\code{\textgreater{} antennaReset}

\code{\textgreater{} setupCCB}
\end{quote}

\item {} 
Select the active surface shape (Shaped configuration for C-band observations)
\begin{quote}

\code{\textgreater{} asSetup=S}
\end{quote}

\item {} 
Insert the Local Oscillator value in MHz
\begin{quote}

\code{\textgreater{} setLO={[}freq{]}}
\end{quote}

\item {} 
Select the Total Power backend
\begin{quote}

\code{\textgreater{} chooseBackend=BACKENDS/TotalPower}
\end{quote}

\item {} 
Insert the bandwidth (300, 730, 1250 or 2000 MHz) and choose the sample rate (in MHz) :
\begin{quote}

\code{\textgreater{} setSection=0,*, {[}bw{]},*,*,{[}sampleRate{]},*}

\code{\textgreater{} setSection=1,*, {[}bw{]},*,*,{[}sampleRate{]},*}
\end{quote}

\item {} 
Put the antenna at 45 deg of elevation and attenuate the signal in order to obtain values between 750 and 1100 counts (linear range of the backend) :
\begin{quote}

\code{\textgreater{} goTo=*,45d}

\code{\textgreater{} getTpi}

\code{\textgreater{} setAttenuation=0,{[}att{]}}        with {[}att{]} between 0 and 15 dB

\code{\textgreater{} setAttenuation=1,{[}att{]}}        with {[}att{]} between 0 and 15 dB

\code{\textgreater{} getTpi}
\end{quote}

\item {} 
Check the tsys (typical values)
\begin{quote}

\code{\textgreater{} tsys}
\end{quote}

\item {} 
Begin the schedule by indicating the start scan {[}N{]} or subscan {[}N\_n{]} in the SCD file :
\begin{quote}

\code{\textgreater{} startSchedule={[}projectID{]}/{[}schedulename{]}.scd,{[}N{]}}
\end{quote}

\end{enumerate}


\paragraph{Check during the observations}
\label{Continuum/C-band/TP/check-obs:check-during-the-observations}\label{Continuum/C-band/TP/check-obs::doc}
\$ : commands to insert in a shell

\textgreater{} : commands to insert in the operatorInput panel
\begin{enumerate}
\item {} 
Data are correctly written in the directory
\begin{quote}

On nuraghe-obs2, check that the data are written in your project section:

\code{\$ cd/archive/data/{[}projectID{]}/}
\end{quote}

\item {} 
Quick-look of the data
\begin{quote}

On nuraghe-obs2:

\code{\$ idl}

\code{IDL\textgreater{} .r fitslook}

\code{IDL\textgreater{} fitslook}

Note that in the quick-look plots, the subscans are shown with a short delay            with respect to the real observations.
\end{quote}

\item {} 
Antenna's monitors
\begin{quote}

Check that everything section is green. If a red box appears, put the cursor on         it and look at the error message.

Contact and report the error messages to the responsible of the observation             (observer friend?).
\end{quote}

\item {} 
Panels
\begin{quote}
\begin{itemize}
\item {} 
Scheduler : status OK, green

\end{itemize}

During the tracking, @ is green while it is red during the slewing of the antenna.

Check the update of the number of scan/subscan according to your schedule. Scans will be skipped if the target is not visible at the            moment of the observations.

If you realize that the scan/subscan number is frozen or that the tracking @ is red while the antenna is tracking the source, stop the          on-going schedule with

\code{\textgreater{} antennaStop}    (in the operatorInput)

then start again the schedule:

\code{\textgreater{} antennaStart={[}projectID{]}/…scd,n}     (with n the number of scan)

{\hspace*{\fill}\includegraphics{{srt_scheduler}.png}\hspace*{\fill}}
\begin{itemize}
\item {} 
AntennaBoss : status OK, green.

\end{itemize}

{\hspace*{\fill}\includegraphics{{srt_antennaboss}.png}\hspace*{\fill}}
\begin{itemize}
\item {} 
Mount : READY, READY, OK green (CHECK!!!!) while the antenna is pointing a source.

\end{itemize}

{\hspace*{\fill}\includegraphics{{srt_mount}.png}\hspace*{\fill}}
\begin{itemize}
\item {} 
MinorServo; tracking @ is green, the status ids OK and green.

\end{itemize}

{\hspace*{\fill}\includegraphics{{srt_minorservo}.png}\hspace*{\fill}}
\begin{itemize}
\item {} 
Receivers : status OK, green.

\end{itemize}

If the derotator (dewar) is used, check the configuration and status (ready green).

{\hspace*{\fill}\includegraphics{{srt_receivers}.png}\hspace*{\fill}}
\end{quote}

\item {} 
Active surface
\begin{quote}

Sometimes, not all of the small squares of the active surface are green. Do not worry for that. Instead, it can be problematic if a             large fraction of the active surface becomes red.
\begin{itemize}
\item {} 
Check that the state of the active surface corresponds to your choice (shaped, shaped fixed, parabolic, parabolic fixed).

\item {} 
“Ok” should be green during the observations.

\end{itemize}

{\hspace*{\fill}\includegraphics{{srt_activesurface}.png}\hspace*{\fill}}
\end{quote}

\item {} 
Log
\begin{quote}
\begin{itemize}
\item {} 
The log file (jlog) contains warning and error messages. Warning messages are indicated in yellow while error messages are in red.

\item {} 
Check the possible error messages. Try to understand the origin of the problem and to solve it. In case of persistent/complex problem, contact and report the error messages and the associated UT to the responsible of the observation (observer friend?).

\end{itemize}

{\hspace*{\fill}\includegraphics{{srt_jlog}.png}\hspace*{\fill}}
\end{quote}

\item {} 
Calibration tool client
\begin{quote}

{\hspace*{\fill}\includegraphics{{srt_calibrationtool}.png}\hspace*{\fill}}
\end{quote}

\item {} 
Weather parameters
\begin{quote}
\begin{quote}

On nuraghe-obs1, activate the meteo client:
\end{quote}

\code{\$ meteoClient}

If the wind speed exceeds 61 km/h, the antenna must be stowed.

{\hspace*{\fill}\includegraphics{{srt_meteo}.png}\hspace*{\fill}}
\end{quote}

\end{enumerate}


\paragraph{Getting your data}
\label{Continuum/C-band/TP/get-data:getting-your-data}\label{Continuum/C-band/TP/get-data::doc}
Your data are in your private directory on \emph{nuraghe-obs2}.

You can download them on your computer whenever you want during the observations.
\begin{quote}

\code{\$ scp -r  {[}projectID{]}@nuraghe-obs2:/archive/data/{[}projectID{]}/* .}
\end{quote}


\paragraph{End of the session}
\label{Continuum/C-band/TP/stop-session:end-of-the-session}\label{Continuum/C-band/TP/stop-session::doc}
Your observations are now finished, we can stop the schedule and park
the antenna.


\subparagraph{On nuraghe-obs1}
\label{Continuum/C-band/TP/stop-session:on-nuraghe-obs1}\begin{enumerate}
\item {} 
Stop your schedule :

\code{\textgreater{} stopSchedule}   \emph{interruption of the current subscan}

\end{enumerate}

or
\begin{quote}

\code{\textgreater{} haltSchedule}    \emph{the schedule stops at the end of the on-going subscan.}
\end{quote}
\begin{enumerate}
\setcounter{enumi}{1}
\item {} 
Park the minor servo, active surface and antenna

\code{\textgreater{} goTo=180d,89d}

\code{\textgreater{} servoPark}

\code{\textgreater{} asPark}

\code{\textgreater{} antennaPark}

\end{enumerate}


\subparagraph{Block the axes of the antenna}
\label{Continuum/C-band/TP/stop-session:block-the-axes-of-the-antenna}
Look at the monitor of the antenna and wait until the upper right
panel becomes red. It can take a few minutes after the command
\emph{antennaPark} has been given.

Only at this moment, you can press on the red button.


\subsubsection{SARDARA}
\label{Continuum/C-band/SARDARA/index:sardara}\label{Continuum/C-band/SARDARA/index::doc}

\paragraph{Before observing}
\label{Continuum/C-band/SARDARA/before-obs:before-observing}\label{Continuum/C-band/SARDARA/before-obs::doc}
Some checks need to be performed before starting the observations.


\subparagraph{On nuraghe-mng}
\label{Continuum/C-band/SARDARA/before-obs:on-nuraghe-mng}\begin{description}
\item[{Check that :}] \leavevmode\begin{itemize}
\item {} 
all of the 31 containers are active on ACS ;

\item {} 
the active surface is green on AS ;

\item {} 
the jlog is open in order to track possible error messages ;

\item {} 
the interface of the Meteo client is open to check the wind velocity in real time (\textless{} 60 km/h).

\end{itemize}

\end{description}


\subparagraph{On nuraghe-obs1}
\label{Continuum/C-band/SARDARA/before-obs:on-nuraghe-obs1}\begin{enumerate}
\item {} 
Check the presence of the 8 panels :
\begin{itemize}
\item {} 
\textbf{operatorInput}

\item {} 
\textbf{AntennaBoss}

\item {} 
\textbf{GenericBackend}

\item {} 
\textbf{Mount}

\item {} 
\textbf{Observatory}

\item {} 
\textbf{Receivers}

\item {} 
\textbf{Scheduler}

\item {} 
\textbf{MinorServo}

\end{itemize}

\end{enumerate}
\begin{enumerate}
\setcounter{enumi}{1}
\item {} 
Upload your shedules (.scd, .lis, .bck and .cfg files) and check them :

\emph{From your computer:}

\code{\$ scp  {[}schedulename.*{]} observer@nuraghe-obs1:/archive/schedules/{[}projectID{]}}

\emph{On nuraghe-obs1:}

\code{\$ cd /archive/schedules/{[}projectID{]}}

\code{\$ scheduleChecker {[}schedulename.scd{]}}

\end{enumerate}


\paragraph{Start the observations}
\label{Continuum/C-band/SARDARA/start-obs:start-the-observations}\label{Continuum/C-band/SARDARA/start-obs::doc}
\$ : commands to insert in a shell

\textgreater{} : commands to insert in the operatorInput panel
\begin{enumerate}
\item {} 
Insert your project number
\begin{quote}

\code{\textgreater{} project={[}projectID{]}}
\end{quote}

\item {} 
Initial setup
\begin{quote}

\code{\textgreater{} antennaReset}

\code{\textgreater{} setupCCB}
\end{quote}

\item {} 
Select the active surface shape (Shaped configuration for C-band observations)
\begin{quote}

\code{\textgreater{} asSetup=S}
\end{quote}

\item {} 
Insert the Local Oscillator value in MHz
\begin{quote}

\code{\textgreater{} setLO={[}freq{]}}
\end{quote}

\item {} 
Select and configure the SARDARA backend
\begin{quote}

\code{\textgreater{} chooseBackend=BACKENDS/Sardara}

\code{\textgreater{} initialize=SC00}
\end{quote}

\item {} 
Set the different parameters of the backend :
\begin{quote}

\code{\textgreater{} setSection={[}sect{]},{[}startFreq{]},{[}bw{]},{[}num-feed{]},{[}polarization{]}, {[}sampleRate{]}, {[}bin{]}}
\begin{description}
\item[{with}] \leavevmode{[}{[}sect{]}=0 in full-stokes observations and {[}sect{]}=0,1 in non full-stokes observations ;{]}
{[}startFreq{]} corresponds to the initial frequency in MHz from the LO value ;
{[}bw{]} the bandwidth in MHz ;
{[}num-feed{]} the number of feed : 1 in C-band ;
{[}polarization{]} the polarization mode (0 or 1 : Left and
Right ; 2 : full-Stokes) ;
{[}sampleRate{]} in MHz;
{[}bin{]} the frequency channels (1024, 2048, 4096, 8192, 16384).

\end{description}
\end{quote}

\item {} 
Choose the integration time in ms (e.g. n=10 corresponds to 100 spectra/sec)
\begin{quote}

\code{\textgreater{} integration={[}n{]}}
\end{quote}

\item {} 
Attenuate the signal based on the rms range {[}-128 ;128{]} and check the value on the interface.
\begin{quote}

\code{\textgreater{} getrms}  ???

\code{\textgreater{} setAttenuation={[}sect{]},{[}att{]}}    with {[}att{]} the attenuation from 0 to 15 dB.
\end{quote}

\item {} 
Check the tsys (typical values)
\begin{quote}

\code{\textgreater{} tsys}
\end{quote}

\item {} 
Begin the schedule by indicating the start scan {[}N{]} or subscan {[}N\_n{]} in the SCD file :
\begin{quote}

\code{\textgreater{} startSchedule={[}projectID{]}/{[}schedulename{]}.scd,{[}N{]}}
\end{quote}

\end{enumerate}


\paragraph{Getting your data}
\label{Continuum/C-band/SARDARA/get-data:getting-your-data}\label{Continuum/C-band/SARDARA/get-data::doc}
Your data are in your private directory on \emph{nuraghe-obs2}.

You can download them on your computer whenever you want during the observations.
\begin{quote}

\code{\$ scp -r observer@dorian:/raid/roach2/* .}  ???

\code{\$ scp -r  {[}projectID{]}@nuraghe-obs2:/archive/data/{[}projectID{]}/*
.} ???
\end{quote}


\paragraph{End of the session}
\label{Continuum/C-band/SARDARA/stop-session:end-of-the-session}\label{Continuum/C-band/SARDARA/stop-session::doc}
Your observations are now finished, we can stop the schedule and park
the antenna.


\subparagraph{On nuraghe-obs1}
\label{Continuum/C-band/SARDARA/stop-session:on-nuraghe-obs1}\begin{enumerate}
\item {} 
Stop your schedule :

\code{\textgreater{} stopSchedule}   \emph{interruption of the current subscan}

\end{enumerate}

or
\begin{quote}

\code{\textgreater{} haltSchedule}    \emph{the schedule stops at the end of the on-going subscan.}
\end{quote}
\begin{enumerate}
\setcounter{enumi}{1}
\item {} 
Park the minor servo, active surface and antenna

\code{\textgreater{} goTo=180d,89d}

\code{\textgreater{} servoPark}

\code{\textgreater{} asPark}

\code{\textgreater{} antennaPark}

\end{enumerate}


\subparagraph{Block the axes of the antenna}
\label{Continuum/C-band/SARDARA/stop-session:block-the-axes-of-the-antenna}
Look at the monitor of the antenna and wait until the upper right
panel becomes red. It can take a few minutes after the command
\emph{antennaPark} has been given.

Only at this moment, you can press on the red button.


\subsection{K-band}
\label{Continuum/K-band/index::doc}\label{Continuum/K-band/index:k-band}

\subsubsection{Total Power}
\label{Continuum/K-band/TP/index:total-power}\label{Continuum/K-band/TP/index::doc}

\paragraph{Before observing}
\label{Continuum/K-band/TP/before-obs:before-observing}\label{Continuum/K-band/TP/before-obs::doc}
Some checks need to be performed before starting the observations.


\subparagraph{On nuraghe-mng}
\label{Continuum/K-band/TP/before-obs:on-nuraghe-mng}\begin{description}
\item[{Check that :}] \leavevmode\begin{itemize}
\item {} 
all of the 31 containers are active on ACS ;

\item {} 
the active surface is green on AS ;

\item {} 
the jlog is open in order to track possible error messages ;

\item {} 
the interface of the Meteo client is open to check the wind velocity in real time (\textless{} 60 km/h).

\end{itemize}

\end{description}


\subparagraph{On nuraghe-obs1}
\label{Continuum/K-band/TP/before-obs:on-nuraghe-obs1}\begin{enumerate}
\item {} 
Check the presence of the 8 panels :
\begin{itemize}
\item {} 
\textbf{operatorInput}

\item {} 
\textbf{AntennaBoss}

\item {} 
\textbf{GenericBackend}

\item {} 
\textbf{Mount}

\item {} 
\textbf{Observatory}

\item {} 
\textbf{Receivers}

\item {} 
\textbf{Scheduler}

\item {} 
\textbf{MinorServo}

\end{itemize}

\end{enumerate}
\begin{enumerate}
\setcounter{enumi}{1}
\item {} 
Upload your shedules (.scd, .lis, .bck and .cfg files) and check them :

\emph{From your computer:}

\code{\$ scp  {[}schedulename.*{]} observer@nuraghe-obs1:/archive/schedules/{[}projectID{]}}

\emph{On nuraghe-obs1:}

\code{\$ cd /archive/schedules/{[}projectID{]}}

\code{\$ scheduleChecker {[}schedulename.scd{]}}

\end{enumerate}


\paragraph{Start the observations}
\label{Continuum/K-band/TP/start-obs:start-the-observations}\label{Continuum/K-band/TP/start-obs::doc}
\$ : commands to insert in a shell

\textgreater{} : commands to insert in the operatorInput panel
\begin{enumerate}
\item {} 
Insert your project number
\begin{quote}

\code{\textgreater{} project={[}projectID{]}}
\end{quote}

\item {} 
Initial setup
\begin{quote}

\code{\textgreater{} antennaReset}

\code{\textgreater{} setupKKG}
\end{quote}

\item {} 
Select the active surface shape (Shaped configuration for K-band observations)
\begin{quote}

\code{\textgreater{} asSetup=S}
\end{quote}

\item {} 
Insert the Local Oscillator value in MHz
\begin{quote}

\code{\textgreater{} setLO={[}freq{]}}
\end{quote}

\item {} 
Select the Total Power backend
\begin{quote}

\code{\textgreater{} chooseBackend=BACKENDS/TotalPower}
\end{quote}

\item {} 
For each section {[}sect{]}, insert the bandwidth ({[}bw{]}=300, 730, 1250 or 2000 MHz) and the sample rate (in MHz) :
\begin{quote}

\code{\textgreater{} setSection={[}sect{]},*, {[}bw{]},*,*,{[}sampleRate{]},*}
\end{quote}

Reminder : in K-band there are 7 feeds, so 14 sections with {[}sect{]}=0,1,2,3,4,5,6,7,8,9,10,11,12,13.

\item {} 
If you want to use the multi-feed derotator to prevent field rotation during long acquisition, select the derotator configuration :
\begin{quote}

\code{\textgreater{} derotatorSetConfiguration={[}config{]}}   with {[}config{]}=BSC, CUSTOM or FIXED
\begin{itemize}
\item {} 
BSC is for Best Coverage Space (automatic rotation of the dewar in order to best cover the scanned area).

\item {} 
CUSTOM : the user has to choose the angle of the dewar axis with

\end{itemize}
\begin{quote}

the y-axis of the scanning frame that will be kept during the
whole duration of the acquisition :
\end{quote}

\code{\textgreater{}  derotatorSetPosition={[}ang{]}d}     with {[}ang{]} the dewar angle in degrees
\begin{itemize}
\item {} 
FIXED : the dewar keeps a fixed postion w.r.t the horizon, no rotation is applied. To specify a static angle :

\end{itemize}

\code{\textgreater{}  derotatorSetPosition={[}ang{]}d}     with {[}ang{]} the dewar angle in degrees
\end{quote}

To read back the position of the dewar :
\begin{quote}

\code{\textgreater{} derotatorGetPosition}
\end{quote}

\item {} 
Put the antenna at 45 deg of elevation and attenuate the signal for

\end{enumerate}
\begin{quote}

the 14 sections {[}sect{]} in order to obtain values between 750 and 1100
counts (linear range of the backend) :
\begin{quote}

\code{\textgreater{} goTo=*,45d}

\code{\textgreater{} getTpi}

\code{\textgreater{} setAttenuation={[}sect{]},{[}att{]}}  with {[}att{]} between 0 and 15 dB

\code{\textgreater{} getTpi}
\end{quote}
\end{quote}
\begin{enumerate}
\item {} 
Check the tsys (typical values)
\begin{quote}

\code{\textgreater{} tsys}
\end{quote}

\item {} 
Begin the schedule by indicating the start scan {[}N{]} or subscan {[}N\_n{]} in the SCD file :
\begin{quote}

\code{\textgreater{} startSchedule={[}projectID{]}/{[}schedulename{]}.scd,{[}N{]}}
\end{quote}

\end{enumerate}


\paragraph{Getting your data}
\label{Continuum/K-band/TP/get-data:getting-your-data}\label{Continuum/K-band/TP/get-data::doc}
Your data are in your private directory on \emph{nuraghe-obs2}.

You can download them on your computer whenever you want during the observations.
\begin{quote}

\code{\$ scp -r  {[}projectID{]}@nuraghe-obs2:/archive/data/{[}projectID{]}/* .}
\end{quote}


\paragraph{End of the session}
\label{Continuum/K-band/TP/stop-session:end-of-the-session}\label{Continuum/K-band/TP/stop-session::doc}
Your observations are now finished, we can stop the schedule and park
the antenna.


\subparagraph{On nuraghe-obs1}
\label{Continuum/K-band/TP/stop-session:on-nuraghe-obs1}\begin{enumerate}
\item {} 
Stop your schedule :

\code{\textgreater{} stopSchedule}   \emph{interruption of the current subscan}

\end{enumerate}

or
\begin{quote}

\code{\textgreater{} haltSchedule}    \emph{the schedule stops at the end of the on-going subscan.}
\end{quote}
\begin{enumerate}
\setcounter{enumi}{1}
\item {} 
Park the minor servo, active surface and antenna

\code{\textgreater{} goTo=180d,89d}

\code{\textgreater{} servoPark}

\code{\textgreater{} asPark}

\code{\textgreater{} antennaPark}

\end{enumerate}


\subparagraph{Block the axes of the antenna}
\label{Continuum/K-band/TP/stop-session:block-the-axes-of-the-antenna}
Look at the monitor of the antenna and wait until the upper right
panel becomes red. It can take a few minutes after the command
\emph{antennaPark} has been given.

Only at this moment, you can press on the red button.


\subsubsection{SARDARA}
\label{Continuum/K-band/SARDARA/index:sardara}\label{Continuum/K-band/SARDARA/index::doc}

\paragraph{Before observing}
\label{Continuum/K-band/SARDARA/before-obs:before-observing}\label{Continuum/K-band/SARDARA/before-obs::doc}
Some checks need to be performed before starting the observations.


\subparagraph{On nuraghe-mng}
\label{Continuum/K-band/SARDARA/before-obs:on-nuraghe-mng}\begin{description}
\item[{Check that :}] \leavevmode\begin{itemize}
\item {} 
all of the 31 containers are active on ACS ;

\item {} 
the active surface is green on AS ;

\item {} 
the jlog is open in order to track possible error messages ;

\item {} 
the interface of the Meteo client is open to check the wind velocity in real time (\textless{} 60 km/h).

\end{itemize}

\end{description}


\subparagraph{On nuraghe-obs1}
\label{Continuum/K-band/SARDARA/before-obs:on-nuraghe-obs1}\begin{enumerate}
\item {} 
Check the presence of the 8 panels :
\begin{itemize}
\item {} 
\textbf{operatorInput}

\item {} 
\textbf{AntennaBoss}

\item {} 
\textbf{GenericBackend}

\item {} 
\textbf{Mount}

\item {} 
\textbf{Observatory}

\item {} 
\textbf{Receivers}

\item {} 
\textbf{Scheduler}

\item {} 
\textbf{MinorServo}

\end{itemize}

\end{enumerate}
\begin{enumerate}
\setcounter{enumi}{1}
\item {} 
Upload your shedules (.scd, .lis, .bck and .cfg files) and check them :

\emph{From your computer:}

\code{\$ scp  {[}schedulename.*{]} observer@nuraghe-obs1:/archive/schedules/{[}projectID{]}}

\emph{On nuraghe-obs1:}

\code{\$ cd /archive/schedules/{[}projectID{]}}

\code{\$ scheduleChecker {[}schedulename.scd{]}}

\end{enumerate}


\paragraph{Start the observations}
\label{Continuum/K-band/SARDARA/start-obs:start-the-observations}\label{Continuum/K-band/SARDARA/start-obs::doc}
\$ : commands to insert in a shell

\textgreater{} : commands to insert in the operatorInput panel
\begin{enumerate}
\item {} 
Insert your project number
\begin{quote}

\code{\textgreater{} project={[}projectID{]}}
\end{quote}

\item {} 
Initial setup
\begin{quote}

\code{\textgreater{} antennaReset}

\code{\textgreater{} setupKKG}
\end{quote}

\item {} 
Select the active surface shape (Shaped configuration for K-band observations)
\begin{quote}

\code{\textgreater{} asSetup=S}
\end{quote}

\item {} 
Insert the Local Oscillator value in MHz
\begin{quote}

\code{\textgreater{} setLO={[}freq{]}}
\end{quote}

\item {} 
Select and configure the SARDARA backend in K-band
\begin{quote}

\code{\textgreater{} chooseBackend=BACKENDS/Sardara}

\code{\textgreater{} initialize={[}code{]}}
\begin{quote}
\begin{description}
\item[{with {[}code{]}=SK00}] \leavevmode{[}central feed only ;{]}
{[}code{]}=SK77 : 7 feeds ;
{[}code{]}=SK03 : feeds 0 and 3 only ;
{[}code{]}=SK06 : feeds 0 and 6 only.

\end{description}
\end{quote}
\end{quote}

\item {} 
Set the different parameters of the backend:

\code{\textgreater{} setSection={[}sect{]},{[}startFreq{]},{[}bw{]},{[}num-feed{]},{[}polarization{]}, {[}sampleRate{]}, {[}bin{]}}
\begin{quote}
\begin{description}
\item[{with}] \leavevmode{[}{[}sect{]}=0,1,2,3,4,5,6 in full-Stokes observations;{]}\begin{description}
\item[{and {[}sect{]}=0,1,2,3,4,5,6,7,8,9,0,11,12,13 in non full-Stokes observations;}] \leavevmode
{[}startFreq{]} corresponds to the initial frequency in MHz from the LO value;
{[}bw{]} the bandwidth in MHz;
{[}num-feed{]} the number of feeds (from 1 to 7);
{[}polarization{]} the polarization mode;
{[}sampleRate{]} in MHz;
{[}bin{]} the frequency channels (1024, 2048, 4096, 8192, 16384).

\end{description}

\end{description}
\end{quote}

\item {} 
Choose the integration time in ms (e.g. n=10 corresponds to 100 spectra/sec)

\code{\textgreater{} integration={[}n{]}}

\item {} 
If you want to use the multi-feed derotator to prevent field rotation during long acquisition, select the derotator configuration :
\begin{quote}

\code{\textgreater{} derotatorSetConfiguration={[}config{]}}   with {[}config{]}=BSC, CUSTOM or FIXED
\begin{itemize}
\item {} 
BSC is for Best Coverage Space (automatic rotation of the dewar in order to best cover the scanned area).

\item {} 
CUSTOM : the user has to choose the angle of the dewar axis with

\end{itemize}
\begin{quote}

the y-axis of the scanning frame that will be kept during the
whole duration of the acquisition :
\end{quote}

\code{\textgreater{}  derotatorSetPosition={[}ang{]}d}     with {[}ang{]} the dewar angle in degrees
\begin{itemize}
\item {} 
FIXED : the dewar keeps a fixed postion w.r.t the horizon, no rotation is applied. To specify a static angle :

\end{itemize}

\code{\textgreater{}  derotatorSetPosition={[}ang{]}d}     with {[}ang{]} the dewar angle in degrees
\end{quote}

To read back the position of the dewar :
\begin{quote}

\code{\textgreater{} derotatorGetPosition}
\end{quote}

\item {} 
Attenuate the signal based on the rms range {[}-128 ;128{]} and check the value on the interface.

\code{\textgreater{} getrms}  ???

\code{\textgreater{} setAttenuation={[}sect{]},{[}att{]}}    with {[}att{]} the attenuation from 0 to 15 dB.

\item {} 
Check the tsys (typical values)
\begin{quote}

\code{\textgreater{} tsys}
\end{quote}

\item {} 
Begin the schedule by indicating the start scan {[}N{]} or subscan {[}N\_n{]} in the SCD file :
\begin{quote}

\code{\textgreater{} startSchedule={[}projectID{]}/{[}schedulename{]}.scd,{[}N{]}}
\end{quote}

\end{enumerate}


\paragraph{Getting your data}
\label{Continuum/K-band/SARDARA/get-data:getting-your-data}\label{Continuum/K-band/SARDARA/get-data::doc}
Your data are in your private directory on \emph{nuraghe-obs2}.

You can download them on your computer whenever you want during the observations.
\begin{quote}

\code{\$ scp -r observer@dorian:/raid/roach2/* .}  ???

\code{\$ scp -r  {[}projectID{]}@nuraghe-obs2:/archive/data/{[}projectID{]}/*
.} ???
\end{quote}


\paragraph{End of the session}
\label{Continuum/K-band/SARDARA/stop-session:end-of-the-session}\label{Continuum/K-band/SARDARA/stop-session::doc}
Your observations are now finished, we can stop the schedule and park
the antenna.


\subparagraph{On nuraghe-obs1}
\label{Continuum/K-band/SARDARA/stop-session:on-nuraghe-obs1}\begin{enumerate}
\item {} 
Stop your schedule :

\code{\textgreater{} stopSchedule}   \emph{interruption of the current subscan}

\end{enumerate}

or
\begin{quote}

\code{\textgreater{} haltSchedule}    \emph{the schedule stops at the end of the on-going subscan.}
\end{quote}
\begin{enumerate}
\setcounter{enumi}{1}
\item {} 
Park the minor servo, active surface and antenna

\code{\textgreater{} goTo=180d,89d}

\code{\textgreater{} servoPark}

\code{\textgreater{} asPark}

\code{\textgreater{} antennaPark}

\end{enumerate}


\subparagraph{Block the axes of the antenna}
\label{Continuum/K-band/SARDARA/stop-session:block-the-axes-of-the-antenna}
Look at the monitor of the antenna and wait until the upper right
panel becomes red. It can take a few minutes after the command
\emph{antennaPark} has been given.

Only at this moment, you can press on the red button.


\subsection{L-band}
\label{Continuum/L-band/index:l-band}\label{Continuum/L-band/index::doc}

\subsubsection{Total Power}
\label{Continuum/L-band/TP/index:total-power}\label{Continuum/L-band/TP/index::doc}

\paragraph{Before observing}
\label{Continuum/L-band/TP/before-obs:before-observing}\label{Continuum/L-band/TP/before-obs::doc}
Some checks need to be performed before starting the observations.


\subparagraph{On nuraghe-mng}
\label{Continuum/L-band/TP/before-obs:on-nuraghe-mng}\begin{description}
\item[{Check that :}] \leavevmode\begin{itemize}
\item {} 
all of the 31 containers are active on ACS ;

\item {} 
the active surface is green on AS ;

\item {} 
the jlog is open in order to track possible error messages ;

\item {} 
the interface of the Meteo client is open to check the wind velocity in real time (\textless{} 60 km/h).

\end{itemize}

\end{description}


\subparagraph{On nuraghe-obs1}
\label{Continuum/L-band/TP/before-obs:on-nuraghe-obs1}\begin{enumerate}
\item {} 
Check the presence of the 8 panels :
\begin{itemize}
\item {} 
\textbf{operatorInput}

\item {} 
\textbf{AntennaBoss}

\item {} 
\textbf{GenericBackend}

\item {} 
\textbf{Mount}

\item {} 
\textbf{Observatory}

\item {} 
\textbf{Receivers}

\item {} 
\textbf{Scheduler}

\item {} 
\textbf{MinorServo}

\end{itemize}

\end{enumerate}
\begin{enumerate}
\setcounter{enumi}{1}
\item {} 
Upload your shedules (.scd, .lis, .bck and .cfg files) and check them :

\emph{From your computer:}

\code{\$ scp  {[}schedulename.*{]} observer@nuraghe-obs1:/archive/schedules/{[}projectID{]}}

\emph{On nuraghe-obs1:}

\code{\$ cd /archive/schedules/{[}projectID{]}}

\code{\$ scheduleChecker {[}schedulename.scd{]}}

\end{enumerate}


\paragraph{Start the observations}
\label{Continuum/L-band/TP/start-obs:start-the-observations}\label{Continuum/L-band/TP/start-obs::doc}
\$ : commands to insert in a shell

\textgreater{} : commands to insert in the operatorInput panel
\begin{enumerate}
\item {} 
Insert your project number
\begin{quote}

\code{\textgreater{} project={[}projectID{]}}
\end{quote}

\item {} 
Initial setup
\begin{quote}

\code{\textgreater{} antennaReset}

\code{\textgreater{} setupLLP}
\end{quote}

\item {} 
Select the receiver Mode :
\begin{quote}

\code{\textgreater{} receiversMode={[}mode{]}}
\begin{quote}
\begin{itemize}
\item {} \begin{description}
\item[{with receiversMode=XXC1;}] \leavevmode
receiversMode=XXC2;
receiversMode=XXC3;
receiversMode=XXC4;
receiversMode=XXC5;
receiversMode=XXL1;
receiversMode=XXL2;
receiversMode=XXL3;
receiversMode=XXL4;
receiversMode=XXL5.

\end{description}

\end{itemize}

Note that C is for \emph{Circular}, L for \emph{Linear polarization} and
1 : all band, 1300-1800 MHz (no filter);
2 : 1320-1780 MHz;
3 : 1350-1450 MHz;
4 : 1300-1800 MHz (band-pass);
5 : 1625-1715 MHz.
\end{quote}
\end{quote}

\item {} 
Select the active surface shape (Parabolic for L-band observations)
\begin{quote}

\code{\textgreater{} asSetup=P}
\end{quote}

\item {} 
Insert the Local Oscillator value in MHz
\begin{quote}

\code{\textgreater{} setLO={[}freq{]}}
\end{quote}

\item {} 
Select the Total Power backend
\begin{quote}

\code{\textgreater{} chooseBackend=BACKENDS/TotalPower}
\end{quote}

\item {} 
Insert the bandwidth for the focus selector (always 2000 MHz in L-band) and choose the sample rate (in MHz) :
\begin{quote}

\code{\textgreater{} setSection=0,*,2000.000000,*,*,{[}sampleRate{]},*}

\code{\textgreater{} setSection=1,*,2000.000000,*,*,{[}sampleRate{]},*}
\end{quote}

\item {} 
Put the antenna at 45 deg of elevation and attenuate the signal in order to obtain values between 750 and 1100 counts (linear range of the backend) :
\begin{quote}

\code{\textgreater{} goTo=*,45d}

\code{\textgreater{} getTpi}

\code{\textgreater{} setAttenuation=0,{[}att{]}}        with {[}att{]} between 0 and 15 dB

\code{\textgreater{} setAttenuation=1,{[}att{]}}        with {[}att{]} between 0 and 15 dB

\code{\textgreater{} getTpi}
\end{quote}

\item {} 
Check the tsys (typical values)
\begin{quote}

\code{\textgreater{} tsys}
\end{quote}

\item {} 
Begin the schedule by indicating the start scan {[}N{]} or subscan {[}N\_n{]} in the SCD file :
\begin{quote}

\code{\textgreater{} startSchedule={[}projectID{]}/{[}schedulename{]}.scd,{[}N{]}}
\end{quote}

\end{enumerate}


\paragraph{Getting your data}
\label{Continuum/L-band/TP/get-data:getting-your-data}\label{Continuum/L-band/TP/get-data::doc}
Your data are in your private directory on \emph{nuraghe-obs2}.

You can download them on your computer whenever you want during the observations.
\begin{quote}

\code{\$ scp -r  {[}projectID{]}@nuraghe-obs2:/archive/data/{[}projectID{]}/* .}
\end{quote}


\paragraph{End of the session}
\label{Continuum/L-band/TP/stop-session:end-of-the-session}\label{Continuum/L-band/TP/stop-session::doc}
Your observations are now finished, we can stop the schedule and park
the antenna.


\subparagraph{On nuraghe-obs1}
\label{Continuum/L-band/TP/stop-session:on-nuraghe-obs1}\begin{enumerate}
\item {} 
Stop your schedule :

\code{\textgreater{} stopSchedule}   \emph{interruption of the current subscan}

\end{enumerate}

or
\begin{quote}

\code{\textgreater{} haltSchedule}    \emph{the schedule stops at the end of the on-going subscan.}
\end{quote}
\begin{enumerate}
\setcounter{enumi}{1}
\item {} 
Park the minor servo, active surface and antenna

\code{\textgreater{} goTo=180d,89d}

\code{\textgreater{} servoPark}

\code{\textgreater{} asPark}

\code{\textgreater{} antennaPark}

\end{enumerate}


\subparagraph{Block the axes of the antenna}
\label{Continuum/L-band/TP/stop-session:block-the-axes-of-the-antenna}
Look at the monitor of the antenna and wait until the upper right
panel becomes red. It can take a few minutes after the command
\emph{antennaPark} has been given.

Only at this moment, you can press on the red button.


\subsubsection{SARDARA}
\label{Continuum/L-band/SARDARA/index:sardara}\label{Continuum/L-band/SARDARA/index::doc}

\paragraph{Before observing}
\label{Continuum/L-band/SARDARA/before-obs:before-observing}\label{Continuum/L-band/SARDARA/before-obs::doc}
Some checks need to be performed before starting the observations.


\subparagraph{On nuraghe-mng}
\label{Continuum/L-band/SARDARA/before-obs:on-nuraghe-mng}\begin{description}
\item[{Check that :}] \leavevmode\begin{itemize}
\item {} 
all of the 31 containers are active on ACS ;

\item {} 
the active surface is green on AS ;

\item {} 
the jlog is open in order to track possible error messages ;

\item {} 
the interface of the Meteo client is open to check the wind velocity in real time (\textless{} 60 km/h).

\end{itemize}

\end{description}


\subparagraph{On nuraghe-obs1}
\label{Continuum/L-band/SARDARA/before-obs:on-nuraghe-obs1}\begin{enumerate}
\item {} 
Check the presence of the 8 panels :
\begin{itemize}
\item {} 
\textbf{operatorInput}

\item {} 
\textbf{AntennaBoss}

\item {} 
\textbf{GenericBackend}

\item {} 
\textbf{Mount}

\item {} 
\textbf{Observatory}

\item {} 
\textbf{Receivers}

\item {} 
\textbf{Scheduler}

\item {} 
\textbf{MinorServo}

\end{itemize}

\end{enumerate}
\begin{enumerate}
\setcounter{enumi}{1}
\item {} 
Upload your shedules (.scd, .lis, .bck and .cfg files) and check them :

\emph{From your computer:}

\code{\$ scp  {[}schedulename.*{]} observer@nuraghe-obs1:/archive/schedules/{[}projectID{]}}

\emph{On nuraghe-obs1:}

\code{\$ cd /archive/schedules/{[}projectID{]}}

\code{\$ scheduleChecker {[}schedulename.scd{]}}

\end{enumerate}


\paragraph{Start the observations}
\label{Continuum/L-band/SARDARA/start-obs:start-the-observations}\label{Continuum/L-band/SARDARA/start-obs::doc}
\$ : commands to insert in a shell

\textgreater{} : commands to insert in the operatorInput panel
\begin{enumerate}
\item {} 
Insert your project number
\begin{quote}

\code{\textgreater{} project={[}projectID{]}}
\end{quote}

\item {} 
Initial setup
\begin{quote}

\code{\textgreater{} antennaReset}

\code{\textgreater{} setupLLP}
\end{quote}

\item {} 
Select the receiver Mode :
\begin{quote}

\code{\textgreater{} receiversMode={[}mode{]}}
\begin{quote}
\begin{itemize}
\item {} \begin{description}
\item[{with receiversMode=XXC1;}] \leavevmode
receiversMode=XXC2;
receiversMode=XXC3;
receiversMode=XXC4;
receiversMode=XXC5;
receiversMode=XXL1;
receiversMode=XXL2;
receiversMode=XXL3;
receiversMode=XXL4;
receiversMode=XXL5.

\end{description}

\end{itemize}

Note that C is for \emph{Circular}, L for \emph{Linear polarization} and
1 : all band, 1300-1800 MHz (no filter);
2 : 1320-1780 MHz;
3 : 1350-1450 MHz;
4 : 1300-1800 MHz (band-pass);
5 : 1625-1715 MHz.
\end{quote}
\end{quote}

\item {} 
Select the active surface shape (Parabolic for L-band observations)
\begin{quote}

\code{\textgreater{} asSetup=P}
\end{quote}

\item {} 
Insert the Local Oscillator value in MHz
\begin{quote}

\code{\textgreater{} setLO={[}freq{]}}
\end{quote}

\item {} 
Select and configure the SARDARA backend in L-band
\begin{quote}

\code{\textgreater{} chooseBackend=BACKENDS/Sardara}

\code{\textgreater{} initialize=SL00}
\end{quote}

\item {} 
Set the different parameters of the backend :
\begin{quote}

\code{\textgreater{} setSection={[}sect{]},{[}startFreq{]},{[}bw{]},{[}num-feed{]},{[}polarization{]}, {[}sampleRate{]}, {[}bin{]}}
\begin{description}
\item[{with}] \leavevmode{[}{[}sect{]}=0 in full-stokes observations and {[}sect{]}=0,1 in non full-stokes observations ;{]}
{[}startFreq{]} corresponds to the initial frequency in MHz from the LO value ;
{[}bw{]} the bandwidth in MHz ;
{[}num-feed{]} the number of feed : 1 in L-band ;
{[}polarization{]} the polarization mode (0 or 1 : Left and
Right ; 2 : full-Stokes) ;
{[}sampleRate{]} in MHz;
{[}bin{]} the frequency channels (1024, 2048, 4096, 8192, 16384).

\end{description}
\end{quote}

\item {} 
Choose the integration time in ms (e.g. n=10 corresponds to 100 spectra/sec)
\begin{quote}

\code{\textgreater{} integration={[}n{]}}
\end{quote}

\item {} 
Attenuate the signal based on the rms range {[}-128 ;128{]} and check the value on the interface.
\begin{quote}

\code{\textgreater{} getrms}  ???

\code{\textgreater{} setAttenuation={[}sect{]},{[}att{]}}    with {[}att{]} the attenuation from 0 to 15 dB.
\end{quote}

\item {} 
Check the tsys (typical values)
\begin{quote}

\code{\textgreater{} tsys}
\end{quote}

\item {} 
Begin the schedule by indicating the start scan {[}N{]} or subscan {[}N\_n{]} in the SCD file :
\begin{quote}

\code{\textgreater{} startSchedule={[}projectID{]}/{[}schedulename{]}.scd,{[}N{]}}
\end{quote}

\end{enumerate}


\paragraph{Getting your data}
\label{Continuum/L-band/SARDARA/get-data:getting-your-data}\label{Continuum/L-band/SARDARA/get-data::doc}
Your data are in your private directory on \emph{nuraghe-obs2}.

You can download them on your computer whenever you want during the observations.
\begin{quote}

\code{\$ scp -r observer@dorian:/raid/roach2/* .}  ???

\code{\$ scp -r  {[}projectID{]}@nuraghe-obs2:/archive/data/{[}projectID{]}/*
.} ???
\end{quote}


\paragraph{End of the session}
\label{Continuum/L-band/SARDARA/stop-session:end-of-the-session}\label{Continuum/L-band/SARDARA/stop-session::doc}
Your observations are now finished, we can stop the schedule and park
the antenna.


\subparagraph{On nuraghe-obs1}
\label{Continuum/L-band/SARDARA/stop-session:on-nuraghe-obs1}\begin{enumerate}
\item {} 
Stop your schedule :

\code{\textgreater{} stopSchedule}   \emph{interruption of the current subscan}

\end{enumerate}

or
\begin{quote}

\code{\textgreater{} haltSchedule}    \emph{the schedule stops at the end of the on-going subscan.}
\end{quote}
\begin{enumerate}
\setcounter{enumi}{1}
\item {} 
Park the minor servo, active surface and antenna

\code{\textgreater{} goTo=180d,89d}

\code{\textgreater{} servoPark}

\code{\textgreater{} asPark}

\code{\textgreater{} antennaPark}

\end{enumerate}


\subparagraph{Block the axes of the antenna}
\label{Continuum/L-band/SARDARA/stop-session:block-the-axes-of-the-antenna}
Look at the monitor of the antenna and wait until the upper right
panel becomes red. It can take a few minutes after the command
\emph{antennaPark} has been given.

Only at this moment, you can press on the red button.


\section{Spectral lines}
\label{SpectralLine/index:spectral-lines}\label{SpectralLine/index::doc}

\subsection{C-band}
\label{SpectralLine/C-band/index::doc}\label{SpectralLine/C-band/index:c-band}

\subsubsection{SARDARA}
\label{SpectralLine/C-band/SARDARA/index:sardara}\label{SpectralLine/C-band/SARDARA/index::doc}

\paragraph{Before observing}
\label{SpectralLine/C-band/SARDARA/before-obs:before-observing}\label{SpectralLine/C-band/SARDARA/before-obs::doc}
Some checks need to be performed before starting the observations.


\subparagraph{On nuraghe-mng}
\label{SpectralLine/C-band/SARDARA/before-obs:on-nuraghe-mng}\begin{description}
\item[{Check that :}] \leavevmode\begin{itemize}
\item {} 
all of the 31 containers are active on ACS ;

\item {} 
the active surface is green on AS ;

\item {} 
the jlog is open in order to track possible error messages ;

\item {} 
the interface of the Meteo client is open to check the wind velocity in real time (\textless{} 60 km/h).

\end{itemize}

\end{description}


\subparagraph{On nuraghe-obs1}
\label{SpectralLine/C-band/SARDARA/before-obs:on-nuraghe-obs1}\begin{enumerate}
\item {} 
Check the presence of the 8 panels :
\begin{itemize}
\item {} 
\textbf{operatorInput}

\item {} 
\textbf{AntennaBoss}

\item {} 
\textbf{GenericBackend}

\item {} 
\textbf{Mount}

\item {} 
\textbf{Observatory}

\item {} 
\textbf{Receivers}

\item {} 
\textbf{Scheduler}

\item {} 
\textbf{MinorServo}

\end{itemize}

\end{enumerate}
\begin{enumerate}
\setcounter{enumi}{1}
\item {} 
Upload your shedules (.scd, .lis, .bck and .cfg files) and check them :

\emph{From your computer:}

\code{\$ scp  {[}schedulename.*{]} observer@nuraghe-obs1:/archive/schedules/{[}projectID{]}}

\emph{On nuraghe-obs1:}

\code{\$ cd /archive/schedules/{[}projectID{]}}

\code{\$ scheduleChecker {[}schedulename.scd{]}}

\end{enumerate}


\paragraph{Start the observations}
\label{SpectralLine/C-band/SARDARA/start-obs:start-the-observations}\label{SpectralLine/C-band/SARDARA/start-obs::doc}
\$ : commands to insert in a shell

\textgreater{} : commands to insert in the operatorInput panel
\begin{enumerate}
\item {} 
Insert your project number
\begin{quote}

\code{\textgreater{} project={[}projectID{]}}
\end{quote}

\item {} 
Initial setup
\begin{quote}

\code{\textgreater{} antennaReset}

\code{\textgreater{} setupCCB}
\end{quote}

\item {} 
Select the active surface shape (Shaped configuration for C-band observations)
\begin{quote}

\code{\textgreater{} asSetup=S}
\end{quote}

\item {} 
Insert the Local Oscillator value in MHz
\begin{quote}

\code{\textgreater{} setLO={[}freq{]}}
\end{quote}

\item {} 
Select and configure the SARDARA backend
\begin{quote}

\code{\textgreater{} chooseBackend=BACKENDS/Sardara}

\code{\textgreater{} initialize=SC00}
\end{quote}

\item {} 
Set the different parameters of the backend :
\begin{quote}

\code{\textgreater{} setSection={[}sect{]},{[}startFreq{]},{[}bw{]},{[}num-feed{]},{[}polarization{]}, {[}sampleRate{]}, {[}bin{]}}
\begin{description}
\item[{with}] \leavevmode{[}{[}sect{]}=0 in full-stokes observations and {[}sect{]}=0,1 in non full-stokes observations ;{]}
{[}startFreq{]} corresponds to the initial frequency in MHz from the LO value ;
{[}bw{]} the bandwidth in MHz ;
{[}num-feed{]} the number of feed : 1 in C-band ;
{[}polarization{]} the polarization mode (0 or 1 : Left and
Right ; 2 : full-Stokes) ;
{[}sampleRate{]} in MHz;
{[}bin{]} the frequency channels (1024, 2048, 4096, 8192, 16384).

\end{description}
\end{quote}

\item {} 
Choose the integration time in ms (e.g. n=10 corresponds to 100 spectra/sec)
\begin{quote}

\code{\textgreater{} integration={[}n{]}}
\end{quote}

\item {} 
Attenuate the signal based on the rms range {[}-128 ;128{]} and check the value on the interface.
\begin{quote}

\code{\textgreater{} getrms}  ???

\code{\textgreater{} setAttenuation={[}sect{]},{[}att{]}}    with {[}att{]} the attenuation from 0 to 15 dB.
\end{quote}

\item {} 
Check the tsys (typical values)
\begin{quote}

\code{\textgreater{} tsys}
\end{quote}

\item {} 
Begin the schedule by indicating the start scan {[}N{]} or subscan {[}N\_n{]} in the SCD file :
\begin{quote}

\code{\textgreater{} startSchedule={[}projectID{]}/{[}schedulename{]}.scd,{[}N{]}}
\end{quote}

\end{enumerate}


\paragraph{Getting your data}
\label{SpectralLine/C-band/SARDARA/get-data:getting-your-data}\label{SpectralLine/C-band/SARDARA/get-data::doc}
Your data are in your private directory on \emph{nuraghe-obs2}.

You can download them on your computer whenever you want during the observations.
\begin{quote}

\code{\$ scp -r observer@dorian:/raid/roach2/* .}  ???

\code{\$ scp -r  {[}projectID{]}@nuraghe-obs2:/archive/data/{[}projectID{]}/*
.} ???
\end{quote}


\paragraph{End of the session}
\label{SpectralLine/C-band/SARDARA/stop-session:end-of-the-session}\label{SpectralLine/C-band/SARDARA/stop-session::doc}
Your observations are now finished, we can stop the schedule and park
the antenna.


\subparagraph{On nuraghe-obs1}
\label{SpectralLine/C-band/SARDARA/stop-session:on-nuraghe-obs1}\begin{enumerate}
\item {} 
Stop your schedule :

\code{\textgreater{} stopSchedule}   \emph{interruption of the current subscan}

\end{enumerate}

or
\begin{quote}

\code{\textgreater{} haltSchedule}    \emph{the schedule stops at the end of the on-going subscan.}
\end{quote}
\begin{enumerate}
\setcounter{enumi}{1}
\item {} 
Park the minor servo, active surface and antenna

\code{\textgreater{} goTo=180d,89d}

\code{\textgreater{} servoPark}

\code{\textgreater{} asPark}

\code{\textgreater{} antennaPark}

\end{enumerate}


\subparagraph{Block the axes of the antenna}
\label{SpectralLine/C-band/SARDARA/stop-session:block-the-axes-of-the-antenna}
Look at the monitor of the antenna and wait until the upper right
panel becomes red. It can take a few minutes after the command
\emph{antennaPark} has been given.

Only at this moment, you can press on the red button.


\subsubsection{Xarcos}
\label{SpectralLine/C-band/Xarcos/index:xarcos}\label{SpectralLine/C-band/Xarcos/index::doc}

\paragraph{Before observing}
\label{SpectralLine/C-band/Xarcos/before-obs:before-observing}\label{SpectralLine/C-band/Xarcos/before-obs::doc}
Some checks need to be performed before starting the observations.


\subparagraph{On nuraghe-mng}
\label{SpectralLine/C-band/Xarcos/before-obs:on-nuraghe-mng}\begin{description}
\item[{Check that :}] \leavevmode\begin{itemize}
\item {} 
all of the 31 containers are active on ACS ;

\item {} 
the active surface is green on AS ;

\item {} 
the jlog is open in order to track possible error messages ;

\item {} 
the interface of the Meteo client is open to check the wind velocity in real time (\textless{} 60 km/h).

\end{itemize}

\end{description}


\subparagraph{On nuraghe-obs1}
\label{SpectralLine/C-band/Xarcos/before-obs:on-nuraghe-obs1}\begin{enumerate}
\item {} 
Check the presence of the 8 panels :
\begin{itemize}
\item {} 
\textbf{operatorInput}

\item {} 
\textbf{AntennaBoss}

\item {} 
\textbf{GenericBackend}

\item {} 
\textbf{Mount}

\item {} 
\textbf{Observatory}

\item {} 
\textbf{Receivers}

\item {} 
\textbf{Scheduler}

\item {} 
\textbf{MinorServo}

\end{itemize}

\end{enumerate}
\begin{enumerate}
\setcounter{enumi}{1}
\item {} 
Upload your shedules (.scd, .lis, .bck and .cfg files) and check them :

\emph{From your computer:}

\code{\$ scp  {[}schedulename.*{]} observer@nuraghe-obs1:/archive/schedules/{[}projectID{]}}

\emph{On nuraghe-obs1:}

\code{\$ cd /archive/schedules/{[}projectID{]}}

\code{\$ scheduleChecker {[}schedulename.scd{]}}

\end{enumerate}


\paragraph{Start the observations}
\label{SpectralLine/C-band/Xarcos/start-obs:start-the-observations}\label{SpectralLine/C-band/Xarcos/start-obs::doc}
\$ : commands to insert in a shell

\textgreater{} : commands to insert in the operatorInput panel
\begin{enumerate}
\item {} 
Insert your project number
\begin{quote}

\code{\textgreater{} project={[}projectID{]}}
\end{quote}

\item {} 
Initial setup
\begin{quote}

\code{\textgreater{} antennaReset}

\code{\textgreater{} setupCCB}
\end{quote}

\item {} 
Select the active surface shape (Shaped configuration for C-band observations)
\begin{quote}

\code{\textgreater{} asSetup=S}
\end{quote}

\item {} 
Insert the Local Oscillator value in MHz
\begin{quote}

\code{\textgreater{} setLO={[}freq{]}}
\end{quote}

\item {} 
Select and configure the XARCOS backend in C-band
\begin{quote}

\code{\textgreater{} chooseBackend=BACKENDS/Xbackends}

\code{\textgreater{} initialize=XC00}
\end{quote}

\item {} 
The \emph{initialize} command and the parameters inserted in the

\end{enumerate}
\begin{quote}

schedule (.bck) directly set the full-Stoke mode, frequency,
bandwidth and sample rate. You can check that the backend parameters
are correct, or modify them by using the following command :
\begin{quote}

\code{\textgreater{} setSection={[}sect{]},{[}startFreq{]},{[}bw{]},*,*, {[}sampleRate{]},*}
\begin{quote}
\begin{description}
\item[{with}] \leavevmode{[}{[}sect{]}=0 in full-Stokes observations ;{]}
{[}startFreq{]} corresponds to the initial frequency in the
125-250 MHz range from the LO value ;
{[}bw{]} the bandwidth : 125.0, 62.5, 31.25, 15.625, 7.8125, 3.90625, 1.953125, 0.9765625 or 0.48828125 MHz ;
{[}sampleRate{]} its value (in MHz) must be twice the bandwidth.
\begin{description}
\item[{\code{*}}] \leavevmode{[}indicates the number of feeds, polarization mode and{]}
frequency channels,  respectively. Let the asterix (\code{*}) for Xarcos observations.

\end{description}

\end{description}
\end{quote}
\end{quote}
\end{quote}
\begin{enumerate}
\item {} 
Begin the schedule by indicating the start scan {[}N{]} or subscan {[}N\_n{]} in the SCD file :
\begin{quote}

\code{\textgreater{} startSchedule={[}projectID{]}/{[}schedulename{]}.scd,{[}N{]}}
\end{quote}

\end{enumerate}


\paragraph{Getting your data}
\label{SpectralLine/C-band/Xarcos/get-data:getting-your-data}\label{SpectralLine/C-band/Xarcos/get-data::doc}
Your data are in your private directory on \emph{nuraghe-obs2}.

You can download them on your computer whenever you want during the observations.
\begin{quote}

\code{\$ scp -r  {[}projectID{]}@nuraghe-obs2:/archive/data/{[}projectID{]}/* .}
\end{quote}


\paragraph{End of the session}
\label{SpectralLine/C-band/Xarcos/stop-session:end-of-the-session}\label{SpectralLine/C-band/Xarcos/stop-session::doc}
Your observations are now finished, we can stop the schedule and park
the antenna.


\subparagraph{On nuraghe-obs1}
\label{SpectralLine/C-band/Xarcos/stop-session:on-nuraghe-obs1}\begin{enumerate}
\item {} 
Stop your schedule :

\code{\textgreater{} stopSchedule}   \emph{interruption of the current subscan}

\end{enumerate}

or
\begin{quote}

\code{\textgreater{} haltSchedule}    \emph{the schedule stops at the end of the on-going subscan.}
\end{quote}
\begin{enumerate}
\setcounter{enumi}{1}
\item {} 
Park the minor servo, active surface and antenna

\code{\textgreater{} goTo=180d,89d}

\code{\textgreater{} servoPark}

\code{\textgreater{} asPark}

\code{\textgreater{} antennaPark}

\end{enumerate}


\subparagraph{Block the axes of the antenna}
\label{SpectralLine/C-band/Xarcos/stop-session:block-the-axes-of-the-antenna}
Look at the monitor of the antenna and wait until the upper right
panel becomes red. It can take a few minutes after the command
\emph{antennaPark} has been given.

Only at this moment, you can press on the red button.


\subsection{K-band}
\label{SpectralLine/K-band/index::doc}\label{SpectralLine/K-band/index:k-band}

\subsubsection{SARDARA}
\label{SpectralLine/K-band/SARDARA/index:sardara}\label{SpectralLine/K-band/SARDARA/index::doc}

\paragraph{Before observing}
\label{SpectralLine/K-band/SARDARA/before-obs:before-observing}\label{SpectralLine/K-band/SARDARA/before-obs::doc}
Some checks need to be performed before starting the observations.


\subparagraph{On nuraghe-mng}
\label{SpectralLine/K-band/SARDARA/before-obs:on-nuraghe-mng}\begin{description}
\item[{Check that :}] \leavevmode\begin{itemize}
\item {} 
all of the 31 containers are active on ACS ;

\item {} 
the active surface is green on AS ;

\item {} 
the jlog is open in order to track possible error messages ;

\item {} 
the interface of the Meteo client is open to check the wind velocity in real time (\textless{} 60 km/h).

\end{itemize}

\end{description}


\subparagraph{On nuraghe-obs1}
\label{SpectralLine/K-band/SARDARA/before-obs:on-nuraghe-obs1}\begin{enumerate}
\item {} 
Check the presence of the 8 panels :
\begin{itemize}
\item {} 
\textbf{operatorInput}

\item {} 
\textbf{AntennaBoss}

\item {} 
\textbf{GenericBackend}

\item {} 
\textbf{Mount}

\item {} 
\textbf{Observatory}

\item {} 
\textbf{Receivers}

\item {} 
\textbf{Scheduler}

\item {} 
\textbf{MinorServo}

\end{itemize}

\end{enumerate}
\begin{enumerate}
\setcounter{enumi}{1}
\item {} 
Upload your shedules (.scd, .lis, .bck and .cfg files) and check them :

\emph{From your computer:}

\code{\$ scp  {[}schedulename.*{]} observer@nuraghe-obs1:/archive/schedules/{[}projectID{]}}

\emph{On nuraghe-obs1:}

\code{\$ cd /archive/schedules/{[}projectID{]}}

\code{\$ scheduleChecker {[}schedulename.scd{]}}

\end{enumerate}


\paragraph{Start the observations}
\label{SpectralLine/K-band/SARDARA/start-obs:start-the-observations}\label{SpectralLine/K-band/SARDARA/start-obs::doc}
\$ : commands to insert in a shell

\textgreater{} : commands to insert in the operatorInput panel
\begin{enumerate}
\item {} 
Insert your project number
\begin{quote}

\code{\textgreater{} project={[}projectID{]}}
\end{quote}

\item {} 
Initial setup
\begin{quote}

\code{\textgreater{} antennaReset}

\code{\textgreater{} setupKKG}
\end{quote}

\item {} 
Select the active surface shape (Shaped configuration for K-band observations)
\begin{quote}

\code{\textgreater{} asSetup=S}
\end{quote}

\item {} 
Insert the Local Oscillator value in MHz
\begin{quote}

\code{\textgreater{} setLO={[}freq{]}}
\end{quote}

\item {} 
Select and configure the SARDARA backend in K-band
\begin{quote}

\code{\textgreater{} chooseBackend=BACKENDS/Sardara}

\code{\textgreater{} initialize={[}code{]}}
\begin{quote}
\begin{description}
\item[{with {[}code{]}=SK00}] \leavevmode{[}central feed only ;{]}
{[}code{]}=SK77 : 7 feeds ;
{[}code{]}=SK03 : feeds 0 and 3 only ;
{[}code{]}=SK06 : feeds 0 and 6 only.

\end{description}
\end{quote}
\end{quote}

\item {} 
Set the different parameters of the backend:

\code{\textgreater{} setSection={[}sect{]},{[}startFreq{]},{[}bw{]},{[}num-feed{]},{[}polarization{]}, {[}sampleRate{]}, {[}bin{]}}
\begin{quote}
\begin{description}
\item[{with}] \leavevmode{[}{[}sect{]}=0,1,2,3,4,5,6 in full-Stokes observations;{]}\begin{description}
\item[{and {[}sect{]}=0,1,2,3,4,5,6,7,8,9,0,11,12,13 in non full-Stokes observations;}] \leavevmode
{[}startFreq{]} corresponds to the initial frequency in MHz from the LO value;
{[}bw{]} the bandwidth in MHz;
{[}num-feed{]} the number of feeds (from 1 to 7);
{[}polarization{]} the polarization mode;
{[}sampleRate{]} in MHz;
{[}bin{]} the frequency channels (1024, 2048, 4096, 8192, 16384).

\end{description}

\end{description}
\end{quote}

\item {} 
Choose the integration time in ms (e.g. n=10 corresponds to 100 spectra/sec)

\code{\textgreater{} integration={[}n{]}}

\item {} 
If you want to use the multi-feed derotator to prevent field rotation during long acquisition, select the derotator configuration :
\begin{quote}

\code{\textgreater{} derotatorSetConfiguration={[}config{]}}   with {[}config{]}=BSC, CUSTOM or FIXED
\begin{itemize}
\item {} 
BSC is for Best Coverage Space (automatic rotation of the dewar in order to best cover the scanned area).

\item {} 
CUSTOM : the user has to choose the angle of the dewar axis with

\end{itemize}
\begin{quote}

the y-axis of the scanning frame that will be kept during the
whole duration of the acquisition :
\end{quote}

\code{\textgreater{}  derotatorSetPosition={[}ang{]}d}     with {[}ang{]} the dewar angle in degrees
\begin{itemize}
\item {} 
FIXED : the dewar keeps a fixed postion w.r.t the horizon, no rotation is applied. To specify a static angle :

\end{itemize}

\code{\textgreater{}  derotatorSetPosition={[}ang{]}d}     with {[}ang{]} the dewar angle in degrees
\end{quote}

To read back the position of the dewar :
\begin{quote}

\code{\textgreater{} derotatorGetPosition}
\end{quote}

\item {} 
Attenuate the signal based on the rms range {[}-128 ;128{]} and check the value on the interface.

\code{\textgreater{} getrms}  ???

\code{\textgreater{} setAttenuation={[}sect{]},{[}att{]}}    with {[}att{]} the attenuation from 0 to 15 dB.

\item {} 
Check the tsys (typical values)
\begin{quote}

\code{\textgreater{} tsys}
\end{quote}

\item {} 
Begin the schedule by indicating the start scan {[}N{]} or subscan {[}N\_n{]} in the SCD file :
\begin{quote}

\code{\textgreater{} startSchedule={[}projectID{]}/{[}schedulename{]}.scd,{[}N{]}}
\end{quote}

\end{enumerate}


\paragraph{Getting your data}
\label{SpectralLine/K-band/SARDARA/get-data:getting-your-data}\label{SpectralLine/K-band/SARDARA/get-data::doc}
Your data are in your private directory on \emph{nuraghe-obs2}.

You can download them on your computer whenever you want during the observations.
\begin{quote}

\code{\$ scp -r observer@dorian:/raid/roach2/* .}  ???

\code{\$ scp -r  {[}projectID{]}@nuraghe-obs2:/archive/data/{[}projectID{]}/*
.} ???
\end{quote}


\paragraph{End of the session}
\label{SpectralLine/K-band/SARDARA/stop-session:end-of-the-session}\label{SpectralLine/K-band/SARDARA/stop-session::doc}
Your observations are now finished, we can stop the schedule and park
the antenna.


\subparagraph{On nuraghe-obs1}
\label{SpectralLine/K-band/SARDARA/stop-session:on-nuraghe-obs1}\begin{enumerate}
\item {} 
Stop your schedule :

\code{\textgreater{} stopSchedule}   \emph{interruption of the current subscan}

\end{enumerate}

or
\begin{quote}

\code{\textgreater{} haltSchedule}    \emph{the schedule stops at the end of the on-going subscan.}
\end{quote}
\begin{enumerate}
\setcounter{enumi}{1}
\item {} 
Park the minor servo, active surface and antenna

\code{\textgreater{} goTo=180d,89d}

\code{\textgreater{} servoPark}

\code{\textgreater{} asPark}

\code{\textgreater{} antennaPark}

\end{enumerate}


\subparagraph{Block the axes of the antenna}
\label{SpectralLine/K-band/SARDARA/stop-session:block-the-axes-of-the-antenna}
Look at the monitor of the antenna and wait until the upper right
panel becomes red. It can take a few minutes after the command
\emph{antennaPark} has been given.

Only at this moment, you can press on the red button.


\subsubsection{Xarcos}
\label{SpectralLine/K-band/Xarcos/index:xarcos}\label{SpectralLine/K-band/Xarcos/index::doc}

\paragraph{Before observing}
\label{SpectralLine/K-band/Xarcos/before-obs:before-observing}\label{SpectralLine/K-band/Xarcos/before-obs::doc}
Some checks need to be performed before starting the observations.


\subparagraph{On nuraghe-mng}
\label{SpectralLine/K-band/Xarcos/before-obs:on-nuraghe-mng}\begin{description}
\item[{Check that :}] \leavevmode\begin{itemize}
\item {} 
all of the 31 containers are active on ACS ;

\item {} 
the active surface is green on AS ;

\item {} 
the jlog is open in order to track possible error messages ;

\item {} 
the interface of the Meteo client is open to check the wind velocity in real time (\textless{} 60 km/h).

\end{itemize}

\end{description}


\subparagraph{On nuraghe-obs1}
\label{SpectralLine/K-band/Xarcos/before-obs:on-nuraghe-obs1}\begin{enumerate}
\item {} 
Check the presence of the 8 panels :
\begin{itemize}
\item {} 
\textbf{operatorInput}

\item {} 
\textbf{AntennaBoss}

\item {} 
\textbf{GenericBackend}

\item {} 
\textbf{Mount}

\item {} 
\textbf{Observatory}

\item {} 
\textbf{Receivers}

\item {} 
\textbf{Scheduler}

\item {} 
\textbf{MinorServo}

\end{itemize}

\end{enumerate}
\begin{enumerate}
\setcounter{enumi}{1}
\item {} 
Upload your shedules (.scd, .lis, .bck and .cfg files) and check them :

\emph{From your computer:}

\code{\$ scp  {[}schedulename.*{]} observer@nuraghe-obs1:/archive/schedules/{[}projectID{]}}

\emph{On nuraghe-obs1:}

\code{\$ cd /archive/schedules/{[}projectID{]}}

\code{\$ scheduleChecker {[}schedulename.scd{]}}

\end{enumerate}


\paragraph{Start the observations}
\label{SpectralLine/K-band/Xarcos/start-obs:start-the-observations}\label{SpectralLine/K-band/Xarcos/start-obs::doc}
\$ : commands to insert in a shell

\textgreater{} : commands to insert in the operatorInput panel
\begin{enumerate}
\item {} 
Insert your project number
\begin{quote}

\code{\textgreater{} project={[}projectID{]}}
\end{quote}

\item {} 
Initial setup
\begin{quote}

\code{\textgreater{} antennaReset}

\code{\textgreater{} setupKKG}
\end{quote}

\item {} 
Select the active surface shape (Shaped configuration for K-band observations)
\begin{quote}

\code{\textgreater{} asSetup=S}
\end{quote}

\item {} 
Insert the Local Oscillator value in MHz
\begin{quote}

\code{\textgreater{} setLO={[}freq{]}}
\end{quote}

\item {} 
Select and configure the XARCOS backend in K-band
\begin{quote}

\code{\textgreater{} chooseBackend=BACKENDS/Xbackends}

\code{\textgreater{} initialize={[}code{]}}

with
\begin{itemize}
\item {} 
\code{{[}code{]}=XK00} : central feed only.
4 full-Stokes sections with bandwidths of 62.5 MHz, 8 MHz, 2 MHz
and 0.5 MHz, each having 2048(x4) channels ;

\item {} 
\code{{[}code{]}=XK77} : 7 feeds.
Full-Stokes sections are recorded, each having a 62.5 MHz
bandwidth and 2048(x4) channels ;

\item {} 
\code{{[}code{]}=XK03} : feeds 0 and 3 only.
Each feed produces two full-Stokes sections respectively having
bandwidths of 62.5 MHz and 4 MHz and 2048(x4) channels ;

\item {} 
\code{{[}code{]}=XK06} : feeds 0 and 6 only.
Each feed produces two full-Stokes sections respectively having
bandwidths of 62.5 MHz and 4 MHz and 2048(x4) channels.

\end{itemize}
\end{quote}

\item {} 
The \emph{initialize} command and the parameters inserted in the schedule (.bck) directly set the full-Stoke mode, frequency, bandwidth and sample rate. You can check that the backend parameters are correct, or modify them by using the following command that you have to repeat for each section number {[}sect{]} :
\begin{quote}

\code{\textgreater{} setSection={[}sect{]},{[}startFreq{]},{[}bw{]},*,*, {[}sampleRate{]},*}

with
\begin{itemize}
\item {} 
\code{{[}sect{]}} = 0, 1, 2, 3, 4, 5, 6 in full-Stokes observations ;

\item {} 
\code{{[}startFreq{]}} corresponds to the initial frequency in the
125-250 MHz range from the LO value ;

\item {} 
\code{{[}bw{]}} the bandwidth : 125.0, 62.5, 31.25, 15.625, 7.8125,
3.90625, 1.953125, 0.9765625 or 0.48828125 MHz ;

\item {} 
\code{{[}sampleRate{]}} its value (in MHz) must be twice the bandwidth ;

\item {} 
\code{*} refers to the number of feeds, polarization mode and
frequency channels, respectively. Let is like this.

\end{itemize}
\end{quote}

\item {} 
If you want to use the multi-feed derotator to prevent field rotation during long acquisition, select the derotator configuration :
\begin{quote}

\code{\textgreater{} derotatorSetConfiguration={[}config{]}}    with {[}config{]}=BSC, CUSTOM or FIXED.
\begin{itemize}
\item {} 
BSC is for Best Coverage Space (automatic rotation of the dewar in order to best cover the scanned area).

\item {} 
CUSTOM : the user has to choose the angle of the dewar axis with

\end{itemize}
\begin{quote}

the y-axis of the scanning frame that will be kept during the
whole duration of the acquisition :
\end{quote}

\code{\textgreater{}  derotatorSetPosition={[}ang{]}d}     with {[}ang{]} the dewar angle in degrees
\begin{itemize}
\item {} 
FIXED : the dewar keeps a fixed postion w.r.t the horizon, no
rotation is applied. This configuration is usually selected for
spectral line observations, with an angle of 0 deg. To specify a static angle :

\end{itemize}

\code{\textgreater{}  derotatorSetPosition={[}ang{]}d}     with {[}ang{]} the dewar angle in degrees
\end{quote}

To read back the position of the dewar :
\begin{quote}

\code{\textgreater{} derotatorGetPosition}
\end{quote}

\item {} 
Begin the schedule by indicating the start scan {[}N{]} or subscan {[}N\_n{]} in the SCD file :
\begin{quote}

\code{\textgreater{} startSchedule={[}projectID{]}/{[}schedulename{]}.scd,{[}N{]}}
\end{quote}

\end{enumerate}


\paragraph{Getting your data}
\label{SpectralLine/K-band/Xarcos/get-data:getting-your-data}\label{SpectralLine/K-band/Xarcos/get-data::doc}
Your data are in your private directory on \emph{nuraghe-obs2}.

You can download them on your computer whenever you want during the observations.
\begin{quote}

\code{\$ scp -r  {[}projectID{]}@nuraghe-obs2:/archive/data/{[}projectID{]}/* .}
\end{quote}


\paragraph{End of the session}
\label{SpectralLine/K-band/Xarcos/stop-session:end-of-the-session}\label{SpectralLine/K-band/Xarcos/stop-session::doc}
Your observations are now finished, we can stop the schedule and park
the antenna.


\subparagraph{On nuraghe-obs1}
\label{SpectralLine/K-band/Xarcos/stop-session:on-nuraghe-obs1}\begin{enumerate}
\item {} 
Stop your schedule :

\code{\textgreater{} stopSchedule}   \emph{interruption of the current subscan}

\end{enumerate}

or
\begin{quote}

\code{\textgreater{} haltSchedule}    \emph{the schedule stops at the end of the on-going subscan.}
\end{quote}
\begin{enumerate}
\setcounter{enumi}{1}
\item {} 
Park the minor servo, active surface and antenna

\code{\textgreater{} goTo=180d,89d}

\code{\textgreater{} servoPark}

\code{\textgreater{} asPark}

\code{\textgreater{} antennaPark}

\end{enumerate}


\subparagraph{Block the axes of the antenna}
\label{SpectralLine/K-band/Xarcos/stop-session:block-the-axes-of-the-antenna}
Look at the monitor of the antenna and wait until the upper right
panel becomes red. It can take a few minutes after the command
\emph{antennaPark} has been given.

Only at this moment, you can press on the red button.


\subsection{L-band}
\label{SpectralLine/L-band/index:l-band}\label{SpectralLine/L-band/index::doc}

\subsubsection{SARDARA}
\label{SpectralLine/L-band/SARDARA/index:sardara}\label{SpectralLine/L-band/SARDARA/index::doc}

\paragraph{Before observing}
\label{SpectralLine/L-band/SARDARA/before-obs:before-observing}\label{SpectralLine/L-band/SARDARA/before-obs::doc}
Some checks need to be performed before starting the observations.


\subparagraph{On nuraghe-mng}
\label{SpectralLine/L-band/SARDARA/before-obs:on-nuraghe-mng}\begin{description}
\item[{Check that :}] \leavevmode\begin{itemize}
\item {} 
all of the 31 containers are active on ACS ;

\item {} 
the active surface is green on AS ;

\item {} 
the jlog is open in order to track possible error messages ;

\item {} 
the interface of the Meteo client is open to check the wind velocity in real time (\textless{} 60 km/h).

\end{itemize}

\end{description}


\subparagraph{On nuraghe-obs1}
\label{SpectralLine/L-band/SARDARA/before-obs:on-nuraghe-obs1}\begin{enumerate}
\item {} 
Check the presence of the 8 panels :
\begin{itemize}
\item {} 
\textbf{operatorInput}

\item {} 
\textbf{AntennaBoss}

\item {} 
\textbf{GenericBackend}

\item {} 
\textbf{Mount}

\item {} 
\textbf{Observatory}

\item {} 
\textbf{Receivers}

\item {} 
\textbf{Scheduler}

\item {} 
\textbf{MinorServo}

\end{itemize}

\end{enumerate}
\begin{enumerate}
\setcounter{enumi}{1}
\item {} 
Upload your shedules (.scd, .lis, .bck and .cfg files) and check them :

\emph{From your computer:}

\code{\$ scp  {[}schedulename.*{]} observer@nuraghe-obs1:/archive/schedules/{[}projectID{]}}

\emph{On nuraghe-obs1:}

\code{\$ cd /archive/schedules/{[}projectID{]}}

\code{\$ scheduleChecker {[}schedulename.scd{]}}

\end{enumerate}


\paragraph{Start the observations}
\label{SpectralLine/L-band/SARDARA/start-obs:start-the-observations}\label{SpectralLine/L-band/SARDARA/start-obs::doc}
\$ : commands to insert in a shell

\textgreater{} : commands to insert in the operatorInput panel
\begin{enumerate}
\item {} 
Insert your project number
\begin{quote}

\code{\textgreater{} project={[}projectID{]}}
\end{quote}

\item {} 
Initial setup
\begin{quote}

\code{\textgreater{} antennaReset}

\code{\textgreater{} setupLLP}
\end{quote}

\item {} 
Select the receiver Mode :
\begin{quote}

\code{\textgreater{} receiversMode={[}mode{]}}
\begin{quote}
\begin{itemize}
\item {} \begin{description}
\item[{with receiversMode=XXC1;}] \leavevmode
receiversMode=XXC2;
receiversMode=XXC3;
receiversMode=XXC4;
receiversMode=XXC5;
receiversMode=XXL1;
receiversMode=XXL2;
receiversMode=XXL3;
receiversMode=XXL4;
receiversMode=XXL5.

\end{description}

\end{itemize}

Note that C is for \emph{Circular}, L for \emph{Linear polarization} and
1 : all band, 1300-1800 MHz (no filter);
2 : 1320-1780 MHz;
3 : 1350-1450 MHz;
4 : 1300-1800 MHz (band-pass);
5 : 1625-1715 MHz.
\end{quote}
\end{quote}

\item {} 
Select the active surface shape (Parabolic for L-band observations)
\begin{quote}

\code{\textgreater{} asSetup=P}
\end{quote}

\item {} 
Insert the Local Oscillator value in MHz
\begin{quote}

\code{\textgreater{} setLO={[}freq{]}}
\end{quote}

\item {} 
Select and configure the SARDARA backend in L-band
\begin{quote}

\code{\textgreater{} chooseBackend=BACKENDS/Sardara}

\code{\textgreater{} initialize=SL00}
\end{quote}

\item {} 
Set the different parameters of the backend :
\begin{quote}

\code{\textgreater{} setSection={[}sect{]},{[}startFreq{]},{[}bw{]},{[}num-feed{]},{[}polarization{]}, {[}sampleRate{]}, {[}bin{]}}
\begin{description}
\item[{with}] \leavevmode{[}{[}sect{]}=0 in full-stokes observations and {[}sect{]}=0,1 in non full-stokes observations ;{]}
{[}startFreq{]} corresponds to the initial frequency in MHz from the LO value ;
{[}bw{]} the bandwidth in MHz ;
{[}num-feed{]} the number of feed : 1 in L-band ;
{[}polarization{]} the polarization mode (0 or 1 : Left and
Right ; 2 : full-Stokes) ;
{[}sampleRate{]} in MHz;
{[}bin{]} the frequency channels (1024, 2048, 4096, 8192, 16384).

\end{description}
\end{quote}

\item {} 
Choose the integration time in ms (e.g. n=10 corresponds to 100 spectra/sec)
\begin{quote}

\code{\textgreater{} integration={[}n{]}}
\end{quote}

\item {} 
Attenuate the signal based on the rms range {[}-128 ;128{]} and check the value on the interface.
\begin{quote}

\code{\textgreater{} getrms}  ???

\code{\textgreater{} setAttenuation={[}sect{]},{[}att{]}}    with {[}att{]} the attenuation from 0 to 15 dB.
\end{quote}

\item {} 
Check the tsys (typical values)
\begin{quote}

\code{\textgreater{} tsys}
\end{quote}

\item {} 
Begin the schedule by indicating the start scan {[}N{]} or subscan {[}N\_n{]} in the SCD file :
\begin{quote}

\code{\textgreater{} startSchedule={[}projectID{]}/{[}schedulename{]}.scd,{[}N{]}}
\end{quote}

\end{enumerate}


\paragraph{Getting your data}
\label{SpectralLine/L-band/SARDARA/get-data:getting-your-data}\label{SpectralLine/L-band/SARDARA/get-data::doc}
Your data are in your private directory on \emph{nuraghe-obs2}.

You can download them on your computer whenever you want during the observations.
\begin{quote}

\code{\$ scp -r observer@dorian:/raid/roach2/* .}  ???

\code{\$ scp -r  {[}projectID{]}@nuraghe-obs2:/archive/data/{[}projectID{]}/*
.} ???
\end{quote}


\paragraph{End of the session}
\label{SpectralLine/L-band/SARDARA/stop-session:end-of-the-session}\label{SpectralLine/L-band/SARDARA/stop-session::doc}
Your observations are now finished, we can stop the schedule and park
the antenna.


\subparagraph{On nuraghe-obs1}
\label{SpectralLine/L-band/SARDARA/stop-session:on-nuraghe-obs1}\begin{enumerate}
\item {} 
Stop your schedule :

\code{\textgreater{} stopSchedule}   \emph{interruption of the current subscan}

\end{enumerate}

or
\begin{quote}

\code{\textgreater{} haltSchedule}    \emph{the schedule stops at the end of the on-going subscan.}
\end{quote}
\begin{enumerate}
\setcounter{enumi}{1}
\item {} 
Park the minor servo, active surface and antenna

\code{\textgreater{} goTo=180d,89d}

\code{\textgreater{} servoPark}

\code{\textgreater{} asPark}

\code{\textgreater{} antennaPark}

\end{enumerate}


\subparagraph{Block the axes of the antenna}
\label{SpectralLine/L-band/SARDARA/stop-session:block-the-axes-of-the-antenna}
Look at the monitor of the antenna and wait until the upper right
panel becomes red. It can take a few minutes after the command
\emph{antennaPark} has been given.

Only at this moment, you can press on the red button.


\section{Pulsar observations}
\label{Pulsar/index:pulsar-observations}\label{Pulsar/index::doc}

\chapter{Indices and tables}
\label{index:indices-and-tables}\begin{itemize}
\item {} 
\DUspan{xref,std,std-ref}{genindex}

\item {} 
\DUspan{xref,std,std-ref}{search}

\end{itemize}



\renewcommand{\indexname}{Index}
\printindex
\end{document}
